%newpage
\section{Introduction}
Aircraft noise can cause annoyance and sleep disturbance. Annoyance and
sleep disturbance is currently predicted using indicators based on time-averaged sound
pressure levels. This completely discards the character of the sound. To obtain a better 
representation of annoyance, the audible aircraft sound should be predicted in order to 
determine the impact of the sound on people.

Auralization is a technique to synthesise the aural aspects of an object or
surrounding. Auralization can therefore be used to create audible aircraft sounds that
can be used in listening tests.

\subsection{Aircraft auralisation tool}
An aircraft auralisation tool is being developed that should provide plausible 
auralisations of aircraft noise for typical urban situations where reflections 
and shielding can play an important role.

In the auralisation tool a distinction is made between a source model and a 
propagation model. The source model describes the emission of the aircraft, i.e. 
the spectral content as function of time and angle (directivity). 
The propagation model describes the propagation of the sound from source to receiver 
and currently supports:
\begin{itemize}
 \item spherical spreading resulting in a decrease of sound pressure with increase of source-receiver distance;
 \item Doppler shift due to relative motion between the moving aircraft and the non-moving receiver;
 \item atmospheric absorption due to relaxation processes;
 \item reflections at the ground and facades.% due to a sudden change in impedance;
\end{itemize}
The propagation model is based on geometrical acoustics. The reflections are determined using the Image Source Method.

\subsection{Turbulence}
When listening to aircraft noise, sound level fluctuations caused by atmospheric 
turbulence are clearly audible. Therefore, to create a realistic auralisation of 
aircraft noise, atmospheric turbulence needs to be included.

Due to spatial inhomogeneities of the wind velocity and temperature in the atmosphere, 
acoustic scattering occurs, affecting the transfer function between source and receiver. 
Both these inhomogeneities and the aircraft position are time-dependent, and therefore 
the transfer function varies with time resulting in the audible fluctuations. 

The theory of turbulence is a statistical theory. A statistical theory fits well 
with the physics of turbulence since turbulence is a consequence of instability of 
fluid flow in relation to very small fluctuations in the fluid \cite{Tatarskii1971}. For an auralisation instantaneous values of the sound pressure at the receiver are required.

Artnzen\cite{Arntzen2013a} included a phase fluctuations filter in their aircraft noise simulator to make the ground effect less pronounced.
The filter was based on a Gaussian spectrum of turbulence.

In this paper an initial model is presented to describe fluctuations in the sound pressure due to atmospheric 
turbulence as function of time. A stationary (frozen) atmosphere is assumed where the 
movement of the aircraft alone gives rise to fluctuations.
In contrast to Arntzen\cite{Arntzen2013a}, both amplitude and phase fluctuations are considered. The model is based on a simple model by Daigle et. al.\cite{Daigle1983}.

