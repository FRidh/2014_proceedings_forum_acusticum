\section{Discussion}
Based on a simple theory presented by Daigle et. al. a method is described to include modulations due to turbulence in an auralisation.
The mean square of the fluctuations seem to be slightly smaller than what can be expected from the theory. It is not known why.

An important limitation of the current model is the fact that the turbulence is assumed to be homogeneous and isotropic, which means the height-dependency of the turbulence due to the boundary conditions is ignored.
% \subsection{Height-dependency}
% First of all, by assuming that the turbulence can be considered homogeneous and 
% isotropic the height-dependency of the turbulence due to the boundary conditions is ignored.
The outer length scale of the turbulence is known to increase with height, as can be seen in 
vertical profiles of the outer scale of turbulence obtained through acoustical sounding by Krasnenko et. al.\cite{Krasnenko2013}.

Also, it is assumed that the outer length scale of the turbulence is much smaller than the Fresnel zone. When the aircraft is directly above the receiver and at relatively low altitude, this assumption will not be valid either.
For example, at an altitude of 200 meters and a frequency of 100 Hz the Fresnel zone is 7.6 meters. The outer length scale is however in the order of 10 meters.

Figure \ref{fig:levels} shows the sound pressure level as function of time of a signal modulated using the presented method.
It should be noted that it is a mere coincidence that the standard deviation is similar to that what is expected at such distances.
More calculations and with longer averaging times need to be done to determine what the standard deviation is after saturation of the log-amplitude.

% For longer averaging times the standard deviation of the level fluctuations seems to level out at 9 dB instead.
In a next step k-space PSTD time-domain simulations will be performed as well to determine saturation distances and standard deviations of the level fluctuations after saturation of the log-amplitude.
These simulations should also reveal more information regarding the coherence of the fluctuations between different frequencies.


% \subsection{Limitations of the model}
% \begin{enumerate}
%  \item The turbulent boundary layer is ignored. While turbulence is height dependent a homogeneous and isotropic turbulence is assumed. 
%  \item (However, the fluctuations are mainly determined close to the source. A height dependent outer length scale is therefore perhaps not that important? Needs reference.)
%  \item In certain positions the outer length scale of the turbulence might not be much smaller than the Fresnel zone. 
% %  \item Lack of amplitude saturation.
% %  \item (Saturation of the log-amplitude can be prevented by having a height-dependent modulation depth?)
% \end{enumerate}

